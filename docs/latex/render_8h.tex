\section{render.h File Reference}
\label{render_8h}\index{render.h@{render.h}}
{\tt \#include $<$stdlib.h$>$}\par
{\tt \#include $<$stdio.h$>$}\par
{\tt \#include $<$string.h$>$}\par
{\tt \#include \char`\"{}scene.h\char`\"{}}\par
{\tt \#include \char`\"{}gui.h\char`\"{}}\par
{\tt \#include \char`\"{}mslio.h\char`\"{}}\par
{\tt \#include \char`\"{}util.h\char`\"{}}\par
\subsection*{Compounds}
\begin{CompactItemize}
\item 
class {\bf Render}
\begin{CompactList}\small\item\em A rendering class that accepts commands from a {\bf Gui} {\rm (p.\,\pageref{classGui})}, and determines using specific graphics libraries how to draw the results on a screen.\item\end{CompactList}\end{CompactItemize}
\subsection*{Defines}
\begin{CompactItemize}
\item 
\#define {\bf MSL\_\-RENDER\_\-H}
\item 
\#define {\bf RENDERCOLORS}\ 10
\end{CompactItemize}


\subsection{Define Documentation}
\index{render.h@{render.h}!MSL_RENDER_H@{MSL\_\-RENDER\_\-H}}
\index{MSL_RENDER_H@{MSL\_\-RENDER\_\-H}!render.h@{render.h}}
\subsubsection{\setlength{\rightskip}{0pt plus 5cm}\#define MSL\_\-RENDER\_\-H}\label{render_8h_a0}


{\bf Value:}\footnotesize\begin{verbatim}
\end{verbatim}\normalsize 
\index{render.h@{render.h}!RENDERCOLORS@{RENDERCOLORS}}
\index{RENDERCOLORS@{RENDERCOLORS}!render.h@{render.h}}
\subsubsection{\setlength{\rightskip}{0pt plus 5cm}\#define RENDERCOLORS\ 10}\label{render_8h_a1}


