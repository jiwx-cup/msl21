\section{Motion Strategy Library Compound List}
Here are the classes, structs, unions and interfaces with brief descriptions:\begin{CompactList}
\item\contentsline{section}{{\bf FDP}  (A dynamic programming approach to nonholonomic planning, as proposed by Barraquand, Latombe, Algorithmica 10:6, pp. 121-155, 1993)}{\pageref{classFDP}}{}
\item\contentsline{section}{{\bf FDPBest\-First}  (Best first search variant, using the Metric in {\bf Problem} {\rm (p.\,\pageref{classProblem})})}{\pageref{classFDPBestFirst}}{}
\item\contentsline{section}{{\bf FDPBi}  (A bidirectional version of forward dynamic programming)}{\pageref{classFDPBi}}{}
\item\contentsline{section}{{\bf FDPStar}  (An A-Star search variant. The Metric in {\bf Problem} {\rm (p.\,\pageref{classProblem})} is used as the cost)}{\pageref{classFDPStar}}{}
\item\contentsline{section}{{\bf FXApp} }{\pageref{classFXApp}}{}
\item\contentsline{section}{{\bf FXDialog\-Box} }{\pageref{classFXDialogBox}}{}
\item\contentsline{section}{{\bf FXMain\-Window} }{\pageref{classFXMainWindow}}{}
\item\contentsline{section}{{\bf Geom}  (Geometric models and collision detection methods)}{\pageref{classGeom}}{}
\item\contentsline{section}{{\bf Geom\-None}  (A class with no geometry -- a collision never happens)}{\pageref{classGeomNone}}{}
\item\contentsline{section}{{\bf Geom\-PQP}  (Parent class PQP-based {\bf list} {\rm (p.\,\pageref{classlist})} of Triangle models)}{\pageref{classGeomPQP}}{}
\item\contentsline{section}{{\bf Geom\-PQP2D}  (A parent class for 2D PQP geometries)}{\pageref{classGeomPQP2D}}{}
\item\contentsline{section}{{\bf Geom\-PQP2DRigid}  (2D rigid body)}{\pageref{classGeomPQP2DRigid}}{}
\item\contentsline{section}{{\bf Geom\-PQP2DRigid\-Multi}  (A collection of 2D rigid bodies)}{\pageref{classGeomPQP2DRigidMulti}}{}
\item\contentsline{section}{{\bf Geom\-PQP3DRigid}  (3D rigid body)}{\pageref{classGeomPQP3DRigid}}{}
\item\contentsline{section}{{\bf Geom\-PQP3DRigid\-Multi}  (A collection of 3D rigid modies)}{\pageref{classGeomPQP3DRigidMulti}}{}
\item\contentsline{section}{{\bf Gui}  (A generic class for designing graphical user interfaces (GUIs))}{\pageref{classGui}}{}
\item\contentsline{section}{{\bf Gui\-Planner} }{\pageref{classGuiPlanner}}{}
\item\contentsline{section}{{\bf Image}  (Used for texture mapping as part of {\bf Render\-GL} {\rm (p.\,\pageref{classRenderGL})})}{\pageref{classImage}}{}
\item\contentsline{section}{{\bf Incremental\-Planner} }{\pageref{classIncrementalPlanner}}{}
\item\contentsline{section}{{\bf list} }{\pageref{classlist}}{}
\item\contentsline{section}{{\bf Model}  (The incremental simulator model)}{\pageref{classModel}}{}
\item\contentsline{section}{{\bf Model1D}  (A simple one-dimensional model for dynamics studies)}{\pageref{classModel1D}}{}
\item\contentsline{section}{{\bf Model2D}  (Base for all 2D models)}{\pageref{classModel2D}}{}
\item\contentsline{section}{{\bf Model2DPoint}  (A point robot in a 2D world)}{\pageref{classModel2DPoint}}{}
\item\contentsline{section}{{\bf Model2DPoint\-Car}  (A point car-like robot in a 2D world)}{\pageref{classModel2DPointCar}}{}
\item\contentsline{section}{{\bf Model2DRigid}  (A holonomic rigid robot in a 2D world)}{\pageref{classModel2DRigid}}{}
\item\contentsline{section}{{\bf Model2DRigid\-Car}  (A rigid car-like robot in a 2D world)}{\pageref{classModel2DRigidCar}}{}
\item\contentsline{section}{{\bf Model2DRigid\-Car\-Forward}  (A rigid car-like robot that can only go forward in a 2D world)}{\pageref{classModel2DRigidCarForward}}{}
\item\contentsline{section}{{\bf Model2DRigid\-Car\-Smooth}  (A rigid car-like robot with continuous steering angles This model is used by Th. Fraichard, Scheuer, Laugier)}{\pageref{classModel2DRigidCarSmooth}}{}
\item\contentsline{section}{{\bf Model2DRigid\-Car\-Smooth2Trailers}  (A rigid car-like robot with continuous steering angles and two trailers)}{\pageref{classModel2DRigidCarSmooth2Trailers}}{}
\item\contentsline{section}{{\bf Model2DRigid\-Car\-Smooth3Trailers}  (A rigid car-like robot with continuous steering angles and three trailers)}{\pageref{classModel2DRigidCarSmooth3Trailers}}{}
\item\contentsline{section}{{\bf Model2DRigid\-Car\-Smooth\-Trailer}  (A rigid car-like robot with continuous steering angles and a trailer. The trailer models are taken from Murray and Sastry, Trans. Automatic Control, Vol 38, No 5, 1993, pp. 700-716)}{\pageref{classModel2DRigidCarSmoothTrailer}}{}
\item\contentsline{section}{{\bf Model2DRigid\-Chain}  (A 2D kinematic chain of bodies)}{\pageref{classModel2DRigidChain}}{}
\item\contentsline{section}{{\bf Model2DRigid\-Dyncar}  (A 5DOF dynamical model of a rigid car. This model uses a linear tire model, which is far from reality. The model was donated by Jim Bernard)}{\pageref{classModel2DRigidDyncar}}{}
\item\contentsline{section}{{\bf Model2DRigid\-Dyncar\-Ntire}  (A 5DOF dynamical model of a rigid car. This model uses a nonlinear tire model. The model was donated by Jim Bernard)}{\pageref{classModel2DRigidDyncarNtire}}{}
\item\contentsline{section}{{\bf Model2DRigid\-Lander}  (A rigid body with two small side thrusters, and a larger lower thruster. The goal is to navigate and softly \char`\"{}land\char`\"{} the craft by firing thrusters, in spite of gravity)}{\pageref{classModel2DRigidLander}}{}
\item\contentsline{section}{{\bf Model2DRigid\-Multi}  (A collection of free-floating bodies in a 2D world)}{\pageref{classModel2DRigidMulti}}{}
\item\contentsline{section}{{\bf Model3D}  (A base class for all models in 3D worlds)}{\pageref{classModel3D}}{}
\item\contentsline{section}{{\bf Model3DDyn}  (A spacecraft model with three thrusters providing both tranlation force and rotation torque)}{\pageref{classModel3DDyn}}{}
\item\contentsline{section}{{\bf Model3DRigid}  (A rigid robot in a 3D world)}{\pageref{classModel3DRigid}}{}
\item\contentsline{section}{{\bf Model3DRigid\-Chain}  (A 3D kinematic chain of bodies that uses DH parameters)}{\pageref{classModel3DRigidChain}}{}
\item\contentsline{section}{{\bf Model3DRigid\-Helical}  (A rigid robot that moves along helical paths in a 3D world)}{\pageref{classModel3DRigidHelical}}{}
\item\contentsline{section}{{\bf Model3DRigid\-Multi}  (A collection of free-floating bodies in a 3D world)}{\pageref{classModel3DRigidMulti}}{}
\item\contentsline{section}{{\bf Model3DRigid\-Tree}  (A 3D kinematic tree of bodies that uses DH parameters)}{\pageref{classModel3DRigidTree}}{}
\item\contentsline{section}{{\bf Model\-Car}  (The same model as {\bf Model2DRigid\-Car} {\rm (p.\,\pageref{classModel2DRigidCar})})}{\pageref{classModelCar}}{}
\item\contentsline{section}{{\bf Model\-Car\-Dyn}  (The same model as {\bf Model2DRigid\-Dyncar} {\rm (p.\,\pageref{classModel2DRigidDyncar})})}{\pageref{classModelCarDyn}}{}
\item\contentsline{section}{{\bf Model\-Car\-Dyn\-Ntire}  (The same model as {\bf Model2DRigid\-Dyncar\-Ntire} {\rm (p.\,\pageref{classModel2DRigidDyncarNtire})})}{\pageref{classModelCarDynNtire}}{}
\item\contentsline{section}{{\bf Model\-Car\-Dyn\-Rollover}  (A car model considering the rolling effect and the pressure on different tires of the car is different. If the pressure on one tire is 0, the car is considered rolling over. The pressure model of the tire is rigid such that pressure can change at instant time, which means: (1) It might be the reason that only forward {\bf RRT} {\rm (p.\,\pageref{classRRT})} tree works. (2) In the Select\-Input function, pressure has to be restored when to test new inputs)}{\pageref{classModelCarDynRollover}}{}
\item\contentsline{section}{{\bf Model\-Car\-Dyn\-Smooth\-Rollover}  (One more dimension than {\bf Model\-Car\-Dyn\-Rollover} {\rm (p.\,\pageref{classModelCarDynRollover})} considering the steering angle can only change continuously)}{\pageref{classModelCarDynSmoothRollover}}{}
\item\contentsline{section}{{\bf Model\-Car\-Smooth}  (The same model as {\bf Model2DRigid\-Car\-Smooth} {\rm (p.\,\pageref{classModel2DRigidCarSmooth})})}{\pageref{classModelCarSmooth}}{}
\item\contentsline{section}{{\bf Model\-Linear}  (A linear systems model: xdot = Ax + Bu)}{\pageref{classModelLinear}}{}
\item\contentsline{section}{{\bf Model\-ND}  (Simple axis-parallel motions in an N-dimensional space)}{\pageref{classModelND}}{}
\item\contentsline{section}{{\bf Model\-Nintegrator}  (The \char`\"{}nonholonomic integrator\char`\"{}, used by R. Brockett and many others)}{\pageref{classModelNintegrator}}{}
\item\contentsline{section}{{\bf MSLEdge} }{\pageref{classMSLEdge}}{}
\item\contentsline{section}{{\bf msl\-GLFace}  (An internal class, used only for {\bf Render\-GL} {\rm (p.\,\pageref{classRenderGL})})}{\pageref{classmslGLFace}}{}
\item\contentsline{section}{{\bf msl\-GLMaterial}  (An internal class, used only for {\bf Render\-GL} {\rm (p.\,\pageref{classRenderGL})})}{\pageref{classmslGLMaterial}}{}
\item\contentsline{section}{{\bf msl\-GLObject}  (An internal class, used only for {\bf Render\-GL} {\rm (p.\,\pageref{classRenderGL})})}{\pageref{classmslGLObject}}{}
\item\contentsline{section}{{\bf MSLGraph} }{\pageref{classMSLGraph}}{}
\item\contentsline{section}{{\bf MSLList} }{\pageref{classMSLList}}{}
\item\contentsline{section}{{\bf MSLMatrix} }{\pageref{classMSLMatrix}}{}
\item\contentsline{section}{{\bf MSLNode} }{\pageref{classMSLNode}}{}
\item\contentsline{section}{{\bf MSLNode\-Greater}  (This is a comparison object to be used for STL-based sorting)}{\pageref{classMSLNodeGreater}}{}
\item\contentsline{section}{{\bf MSLNode\-Info}  (The information holded in this class is explained in \char`\"{}Reducing Metric Sensitivity in Randomized Trajectory Design\char`\"{} in IEEE/RSJ International Conference on Intelligent Robots and Systems, 2001)}{\pageref{classMSLNodeInfo}}{}
\item\contentsline{section}{{\bf MSLNode\-Less}  (This is a comparison object to be used for STL-based sorting)}{\pageref{classMSLNodeLess}}{}
\item\contentsline{section}{{\bf MSLPlanner\-Window} }{\pageref{classMSLPlannerWindow}}{}
\item\contentsline{section}{{\bf MSLPlot\-Window} }{\pageref{classMSLPlotWindow}}{}
\item\contentsline{section}{{\bf MSLPoint} }{\pageref{classMSLPoint}}{}
\item\contentsline{section}{{\bf MSLPoint3d} }{\pageref{classMSLPoint3d}}{}
\item\contentsline{section}{{\bf MSLPolygon} }{\pageref{classMSLPolygon}}{}
\item\contentsline{section}{{\bf MSLRandom\-Source} }{\pageref{classMSLRandomSource}}{}
\item\contentsline{section}{{\bf MSLTree} }{\pageref{classMSLTree}}{}
\item\contentsline{section}{{\bf MSLTriangle}  (A 3D triangle, made of 3 3D points)}{\pageref{classMSLTriangle}}{}
\item\contentsline{section}{{\bf MSLVector} }{\pageref{classMSLVector}}{}
\item\contentsline{section}{{\bf MSLVertex} }{\pageref{classMSLVertex}}{}
\item\contentsline{section}{{\bf MSLVertex\-Greater}  (This is a comparison object to be used for STL-based sorting)}{\pageref{classMSLVertexGreater}}{}
\item\contentsline{section}{{\bf MSLVertex\-Less}  (This is a comparison object to be used for STL-based sorting)}{\pageref{classMSLVertexLess}}{}
\item\contentsline{section}{{\bf Multi\-Array}  (A multidimensional array made from a 1D vector)}{\pageref{classMultiArray}}{}
\item\contentsline{section}{{\bf Planner}  (The base class for all path planners)}{\pageref{classPlanner}}{}
\item\contentsline{section}{{\bf PRM}  (A probabilistic roadmap planner, proposed by Kavraki, Svestka, Latombe, Overmars, 1994)}{\pageref{classPRM}}{}
\item\contentsline{section}{{\bf Problem}  (An interface class that provides the primary input to a planner)}{\pageref{classProblem}}{}
\item\contentsline{section}{{\bf Random\-Tree}  (Naively extend the tree by random node selection (not really an {\bf RRT} {\rm (p.\,\pageref{classRRT})}))}{\pageref{classRandomTree}}{}
\item\contentsline{section}{{\bf RCRRT}  (Resolution Complete Rapidly-Exploring Random Trees , by Peng Cheng and Steven M. La\-Valle, submitted to 2002 IEEE International Conference on Robotics and Automation. Techniques applied to improve the performance: (1) Combining systematic search with random search such that it has both the completeness of the systematic search and fast searching of the random search. (2) Constraint violation tendency to avoid obstacles This basic planner is used to do the experiment with dynamic car model in the virtual town. The rolling effect of the car and the nonlinear tire model are considered in the model)}{\pageref{classRCRRT}}{}
\item\contentsline{section}{{\bf RCRRTBall}  ({\bf RCRRT} {\rm (p.\,\pageref{classRCRRT})} planner using ball neighborhood to exclude the repeated states)}{\pageref{classRCRRTBall}}{}
\item\contentsline{section}{{\bf RCRRTBall\-Dual}  (Basic dual tree version of {\bf RCRRTBall} {\rm (p.\,\pageref{classRCRRTBall})})}{\pageref{classRCRRTBallDual}}{}
\item\contentsline{section}{{\bf RCRRTBall\-Ext\-Ext}  (Dual tree version of {\bf RCRRTBall} {\rm (p.\,\pageref{classRCRRTBall})} with Ext\-Ext method)}{\pageref{classRCRRTBallExtExt}}{}
\item\contentsline{section}{{\bf RCRRTDual}  (Basic dual tree version of {\bf RCRRT} {\rm (p.\,\pageref{classRCRRT})})}{\pageref{classRCRRTDual}}{}
\item\contentsline{section}{{\bf RCRRTExt\-Ext}  (Dual tree version of {\bf RCRRT} {\rm (p.\,\pageref{classRCRRT})} with Ext\-Ext method, which is used to do experiments with spacecraft model in 3D grid environment considering the dynamic constraints )}{\pageref{classRCRRTExtExt}}{}
\item\contentsline{section}{{\bf Render}  (A rendering class that accepts commands from a {\bf Gui} {\rm (p.\,\pageref{classGui})}, and determines using specific graphics libraries how to draw the results on a screen)}{\pageref{classRender}}{}
\item\contentsline{section}{{\bf Render\-GL}  (Perform 3D rendering using the GL and GLUT libraries)}{\pageref{classRenderGL}}{}
\item\contentsline{section}{{\bf Render\-Iv}  (Perform 3D rendering using the Open\-Inventor library)}{\pageref{classRenderIv}}{}
\item\contentsline{section}{{\bf Render\-Performer}  (Perform 3D rendering using the SGI IRIS Performer library)}{\pageref{classRenderPerformer}}{}
\item\contentsline{section}{{\bf Roadmap\-Planner} }{\pageref{classRoadmapPlanner}}{}
\item\contentsline{section}{{\bf RRT}  (The base class, which generates a single Rapidly-exploring Random Tree)}{\pageref{classRRT}}{}
\item\contentsline{section}{{\bf RRTBidir\-Balanced}  (Balanced Bidirectional {\bf RRTCon\-Con} {\rm (p.\,\pageref{classRRTConCon})} planner)}{\pageref{classRRTBidirBalanced}}{}
\item\contentsline{section}{{\bf RRTCon}  (Replaces Extend with Connect)}{\pageref{classRRTCon}}{}
\item\contentsline{section}{{\bf RRTCon\-Con}  (Use Connect for both exploration and connecting of trees)}{\pageref{classRRTConCon}}{}
\item\contentsline{section}{{\bf RRTDual}  (Planners that grow trees from the initial and goal)}{\pageref{classRRTDual}}{}
\item\contentsline{section}{{\bf RRTExt\-Con}  (Use Connect instead of Extend to connect the two trees)}{\pageref{classRRTExtCon}}{}
\item\contentsline{section}{{\bf RRTExt\-Ext}  (Balance between growing trees toward each other and exploring)}{\pageref{classRRTExtExt}}{}
\item\contentsline{section}{{\bf RRTGoal\-Bias}  (With some probability, choose the goal instead of a random sample)}{\pageref{classRRTGoalBias}}{}
\item\contentsline{section}{{\bf RRTGoal\-Zoom}  (Bias the samples toward the goal as the tree gets closer)}{\pageref{classRRTGoalZoom}}{}
\item\contentsline{section}{{\bf RRTHull}  (Grow a Rapidly-exploring Random Tree in a large disc)}{\pageref{classRRTHull}}{}
\item\contentsline{section}{{\bf RRTPolar}  (Gradually bias the sampling towards the goal)}{\pageref{classRRTPolar}}{}
\item\contentsline{section}{{\bf RRTSlide}  (In the Connect method, slide along the walls)}{\pageref{classRRTSlide}}{}
\item\contentsline{section}{{\bf Scene}  (An interface class that gives {\bf Problem} {\rm (p.\,\pageref{classProblem})} information to {\bf Render} {\rm (p.\,\pageref{classRender})}. It tells the renderer how the \char`\"{}scene\char`\"{} appears for rendering purposes, as opposed to collision-detection purposes)}{\pageref{classScene}}{}
\item\contentsline{section}{{\bf Solver}  (The base class for all path planners)}{\pageref{classSolver}}{}
\end{CompactList}
